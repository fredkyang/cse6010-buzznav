\documentclass[fleqn,10pt]{olplainarticle}
% Use option lineno for line numbers 
\setlength{\parindent}{0pt}
\setlength{\parskip}{6pt}
\usepackage{xcolor}
\colorlet{color2}{white}  % olplainarticle 使用 color2 作为 abstract 文字颜色
\bibliographystyle{apalike}
\bibliographystyle{unsrtnat}   % <- orders entries by citation order
\usepackage[colorlinks=true,linkcolor=blue,citecolor=blue,urlcolor=blue]{hyperref}
\usepackage{subcaption}


\title{CSE 6010 CheckPoint 2: \\Campus Navigation System}

\author[1]{Jing He}
\author[2]{Fred Yang}
\author[3]{Haowen Jiang}
\author[4]{Ming-Cheng Fan}
\author[5]{Chia-Hsin Chiu}
\affil[1]{jhe468@gatech.edu}
\affil[2]{fred.yang@gatech.edu}
\affil[3]{hjiang401@gatech.edu}
\affil[4]{mfan77@gatech.edu}
\affil[5]{cchiu73@gatech.edu}


\usepackage[numbers,sort&compress]{natbib}
\makeatletter
\makeatother
\begin{document}

\flushbottom
\maketitle


\section*{Current State}
At present, we have completed the data generation function and produced the required data. Through this function, we constructed the adjacency list of the campus. Since the adjacency list stores the IDs of each node (representing buildings or roads), we also generated a mapping table between building names and their corresponding node IDs.

\section*{Initial Results}
The file \texttt{building\_mapping.csv} serves as a mapping table between building names and their corresponding node IDs in the campus road network. In total, the file consists of \textbf{384 rows}, meaning that 384 buildings are mapped to their respective graph nodes.It contains two columns:

\begin{itemize}
    \item \textbf{building\_name}: the name of each building (e.g., \textit{Baker Building}, \textit{John Lewis Student Center});
    \item \textbf{node\_id}: the unique identifier of the corresponding node in the road network graph.
\end{itemize}

The file \texttt{adj\_list.csv} stores the adjacency list representation of the campus road network.In total, the file contains \textbf{1,933 rows}, each corresponding to an edge in the network. It contains three columns:

\begin{itemize}
    \item \textbf{src}: the source node ID, representing the starting point of an edge in the graph;
    \item \textbf{dst}: the destination node ID, representing the endpoint of the same edge;
    \item \textbf{length}: the physical distance (in meters) between the source and destination nodes.
\end{itemize}

\begin{table}[h]
\centering
\begin{minipage}[t]{0.45\textwidth}
\centering
\textbf{(a) building\_map.csv}\\[4pt]
\begin{tabular}{l c}
\hline
\textbf{building\_name} & \textbf{node\_id} \\
\hline
Ferst Drive \& State Street & 727 \\
Klaus Building & 728 \\
Fitten Hall & 729 \\
Techwood Drive \& North Avenue & 730 \\
\hline
\end{tabular}
\end{minipage}
\hfill
\begin{minipage}[t]{0.45\textwidth}
\centering
\textbf{(b) adj\_list.csv}\\[4pt]
\begin{tabular}{c c c}
\hline
\textbf{src} & \textbf{dst} & \textbf{length (m)} \\
\hline
172 & 173 & 16.5337660224793 \\
172 & 983 & 1.8516020234408266 \\
173 & 984 & 12.801718407406037 \\
173 & 1047 & 7.146090258556631 \\
\hline
\end{tabular}
\end{minipage}
\caption[b]{Examples of \texttt{building\_map.csv} and \texttt{adj\_list.csv}}
\end{table}


\section*{Division of labor}
\begin{table}[h]
  \centering
  \begin{tabular}{l l l c}
    \toprule
    \textbf{Task} & \textbf{People} & \textbf{Start Date} & \textbf{Duration} \\
    \midrule
    Task & All & Sep 10 & 1 week \\
    Report Writing & All & Nov 26 & 2 weeks \\
    \bottomrule
  \end{tabular}
  \caption{High-level schedule for Scholar Compass}
  \label{tab:schedule}
  \vspace{0ex}
  \noindent\textit{All team members have contributed a similar amount of effort to this proposal and will maintain an equitable distribution of work.}
\end{table}
% Last Page
\newpage
\section*{Acknowledgment}

We thank the Georgia Tech CSE 6010 teaching staff for their guidance during the early stages of this project.  
The current implementation and progress can be found at:  
\url{https://github.com/fredkyang/gt_cse6010_navigation_system}

This repository is intended solely for use in CSE 6010 (Fall 25) coursework.

About AI: literature review is summarized by AI.
\bibliographystyle{unsrt}
\bibliography{references}

\end{document}